% the document class specification for the proposal writing process, add the 'submit' option
% for submitting (switches off various draft features); add the 'public' option to exclude
% any private parts. 
\documentclass[RAM]{dfgproposal}
%\documentclass[submit]{dfgproposal}
%\documentclass[submit,public]{dfgproposal}
\bibliography{./lib/dummy}
\usepackage[utf8]{inputenc}

% the following lines get updated by subversion keyword replacement. They are used by the 
% \svninfo package in draft mode to generate metadata. 
\svnInfo $Id: proposal.tex 22963 2012-01-13 08:47:33Z kohlhase $
\svnKeyword $HeadURL: https://svn.kwarc.info/repos/kwarc/doc/macros/forCTAN/proposal/dfg/examples/proposal/proposal.tex $
%
\WAperson[id=miko, 
           personaltitle=Prof. Dr.,
           birthdate=13. September 1964,
           academictitle=Professor of Computer Science,
           affiliation=jacu,
           department=case,
           privaddress=None of your business,
           privtel=that neither,
           email=m.kohlhase@jacobs-university.de,
           workaddress={Campus Ring 1, 28757 Bremen},
           worktel=+49 421 200 3140,
           worktelfax=+49 421 200 3140/493140,
           workfax=+49 421 200 493140]
           {Michael Kohlhase}

\WAperson[id=gc,
           personaltitle=Dr.,
           academictitle=Senior Researcher,
           birthdate=14. April 1972,
           affiliation=pcg,
           department=pcsa,
           privaddress=None of your business,
           privtel=that neither,
           workaddress={PCG Way 7, Hooville},
           worktel=+49 421 0815 4711,
           workfax=+49 421 0815 4712,
           email=gc@pcg.phony]
           {Great Communicator}

\WAinstitution[id=case,acronym=CASE,shortname=CASE,
                url=http://jacobs-university.de/ses/case,
                partof=jacu]
               {Center for Advanced Systems Engineering}

\WAinstitution[id=jacu,acronym=JacU,
               url=http://jacobs-university.de,
               streetaddress={Campus Ring 1},
               townzip={28759 Bremen},
               countryshort=D,
               country=Germany,
               type=University,
               logo=jacobs-logo.png,
               shortname=Jacobs University]
               {Jacobs University Bremen}

\WAinstitution[id=pcsa,
                           url=http://pcg.phony/sa,
                           partof=pcg,shortname=Science Affairs]
               {Science Affairs}
\WAinstitution[id=pcg,acronym=PCG,
                           url=http://pcg.phony,
                           countryshort=D,
                           streetaddress={Seefahrtstrasse 5},
                           townzip={23555 Hamburg},
                           shortname=Power Consulting]
               {Power Consulting GmbH}

%%% Local Variables: 
%%% mode: latex
%%% End: 

% LocalWords:  WAperson miko personaltitle academictitle privaddress privtel
% LocalWords:  workaddress worktel workfax gc



\begin{document}

\begin{center}\color{red}\huge
  This mock proposal is just an example for \texttt{dfgproposal.cls} it reflects the 
  current DFG template valid from October 2011.
\end{center}

\urldef{\gcpubs}\url{http://www.pcg.phony/~gc/pubs.html}
\urldef{\mikopubs}\url{http://kwarc.info/kohlhase/publications.html}
\begin{proposal}[PI=miko,PI=gc,site=jacu,site=pcg,
  pubspage=mikopubs,
  pubspage=gcpubs,
  thema=Intelligentes Schreiben von Antr\"agen,
  acronym={iPoWr},
  acrolong={\underline{I}ntelligent} {\underline{P}r\underline{o}sal} {\underline{Wr}iting},
  title=\pn: \protect\pnlong,
  totalduration=3 years,
  since=1. Feb 2009,
  start=1. Feb. 2010,
  months=24,
  pcgRM=36, pcgRAM=36, jacuRM=36, jacuRAM=36,
  discipline=Computer Science, 
  areas=Knowledge Management]

\begin{Summary}
  \begin{todo}{}
    Summarize the relevant goals of the proposed project in generally intellegible
    terms. Do not use more than 15 lines (max. 1600 characters).
  \end{todo}
  Writing grant proposals is a collaborative effort that requires the integration of
  contributions from many individuals. The use of an ASCII-based format like LaTeX allows
  to coordinate the process via a source code control system like Subversion, allowing the
  proposal writing team to concentrate on the contents rather than the mechanics of
  wrangling with text fragments and revisions.
\end{Summary}

% It is often good to separate the top-level sections into separate files. 
% Especially in collaborative proposals. We do this here. 
\svnInfo $Id: state.tex 22814 2011-12-20 15:00:19Z kohlhase $
\svnKeyword $HeadURL: https://svn.kwarc.info/repos/kwarc/doc/macros/forCTAN/proposal/dfg/examples/proposal/state.tex $

\section{State of the Art and Preliminary Work \deu{(Stand der Forschung und eigene Vorarbeiten)}}\label{stand}

\subsection{List of Project-Related Publications \deu{(Projektbezogenes Publikationsverzeichnis)}}

\begin{todo}{from the proposal template}
  Please include a list of own publications that are related to the proposed project. It
  serves as an important basis for assessing your proposal. The number of publications to
  cite here is determined as follows:
  \begin{compactdesc}
    \item[Single applicant] two publications per year of the funding duration
    \item[Multiple applicants] three publications per year of the funding duration
    \end{compactdesc}
    These rules refer to the proposed funding duration for new proposals and the completed
    duration for renewal proposals.
    
    If you are submitting a proposal to the DFG for the first time and have therefore not
    published in the proposed research area, please list the up to five most important
    publications so far.
\end{todo}

\subsubsection{Peer-Reviewed Articles \deu{(Artikel mit wissenschaftlicher Qualitätssicherung)}}

\dfgprojpapers{Kohlhase:pdpl10,providemore}

\ednote{Anmerkung Jens: Ein nützliches Feature wäre hier, wenn das Paket eine (eventuell
  über Optionen der Dokumentklasse unterdrückbare) Warnung ausgeben würde, wenn zu viele
  Publikationen entsprechend DFG-Richtlinien angegeben werden. Die Anzahl ist sehr eng
  begrenzt.}

\subsubsection{Other Articles \deu{(Andere Artikel)}} None.

\subsubsection{Patents \deu{(Patente)}} None.

%%% Local Variables: 
%%% mode: LaTeX
%%% TeX-master: "proposal"
%%% End: 

% LocalWords:  subsubsections dfgprojpapers pdpl10 providemore compactdesc
% LocalWords:  ourpubs nociteprolist KohKoh ccbssmt09 KohRabZho tmlmrsca10
% LocalWords:  Hutter09 sifemp09

\svnInfo $Id: workplan.tex 23021 2012-01-19 13:30:19Z kohlhase $
\svnKeyword $HeadURL: https://svn.kwarc.info/repos/kwarc/doc/macros/forCTAN/proposal/dfg/examples/proposal/workplan.tex $
\section{Objectives and Work Programme \deu{(Ziele und Arbeitsprogramm)}}

\subsection{Anticipated total duration of the project \deu{(Voraussichtliche Gesamtdauer des Projekts)}}

\begin{todo}{from the proposal template}
Please state
\begin{itemize}
 \item the project's intended duration 1 and how long DFG funds will be necessary,
 \item for ongoing projects: since when the project has been active.
\end{itemize}
\end{todo}

\subsection{Objectives \deu{(Ziele)}}

\begin{objective}[id=firstobj,title=Supporting Authors]
  This is the first objective, after all we have to write proposals all the time, and we
  would rather spend time on research. 
\end{objective}

\begin{objective}[id=secondobj,title=Supporting Reviewers]
  They are only human too, so let's have a heart for them as well. 
\end{objective}


\subsection{Work programme including proposed research methods \deu{(Arbeitsprogramm inkl. vorgesehener Untersuchungsmethoden)}}

%%%%%%%%%%%%%%%%%%%%%%%%%%%%%%%%%%%%%%%%%%%%%%%%%%%%%%%%%%%%%%%%%%%%%%%%%%%%%%%%%
%\subsection{Description (Beschreibung)}%\label{sec:state}
\LaTeX is the best document markup language, it can even be used for literate
programming~\cite{DK:LP,Lamport:ladps94,Knuth:ttb84}
\begin{todo}{from the proposal template}
 review the state of the art in the and your own contribution to it; probably you want to
  divide this into subsubsections. 
\end{todo}

% \dfgprojpapers{Kohlhase:pdpl10,providemore}
\begin{todo}{from the proposal template}
For each applicant

Please give a detailed account of the steps planned during the proposed funding pe-
riod. (For experimental projects, a schedule detailing all planned experiments should
be provided.)

The quality of the work programme is critical to the success of a funding proposal. The
work programme should clearly state how much funding will be requested, why the
funds are needed, and how they will be used, providing details on individual items
where applicable.

Please provide a detailed description of the methods that you plan to use in the project:
What methods are already available? What methods need to be developed? What as-
sistance is needed from outside your own group/institute?
Please list all cited publications pertaining to the description of your work programme
in your bibliography under section 3.
\end{todo}

The project is organized around \pdatacount{all}{wa} large-scale work areas which correspond
to the objectives formulated above. These are subdivided into \pdatacount{all}{wp} work
packages, which we summarize in Figure~\ref{fig:wplist}. Work area
\WAref{mansubsus} will run over the whole project\ednote{come up with a better
  example, this is still oriented towards an EU project} duration of {\pn}. All
{\pdatacount{systems}{wp}} work packages in {\WAref{systems}} will and have to be
covered simultaneously in order to benefit from design-implementation-application feedback
loops.

\wpfig

\begin{workplan}
\begin{workarea}[id=mansubsus,title={Management, Support \& Sustainability}, short=Management]
  This work-group corresponds to Objective \OBJref{firstobj} and has two work packages:
  one for management proper ({\WPref{management}}), and one each for
  dissemination ({\WPref{dissem}})
   
  This work group ensures the dissemination and creation of the periodic integrative
  reports containing the periodic Project Management Report, the Project Management
  Handbook, an Knowledge Dissemination Plan ({\WPref{management}}), the Proceedings of the
  Annual {\pn} Summer School as well as non-public Dissemination and Exploitation plans
  ({\WPref{dissem}}), as well as a report of the {\pn} project milestones.
   
\begin{workpackage}[id=management,lead=jacu,
  title=Project Management,
 jacuRM=2,jacuRAM=8,pcgRM=2]
  Based on the ``Bewilligungsbescheid'' of the DFG, and based on the financial and
  administrative data agreed, the project manager will carry out the overall project
  management, including administrative management.  A project quality handbook will be
  defined, and a {\pn} help-desk for answering questions about the format (first
  project-internal, and after month 12 public) will be established. The project management
  will consist of the following tasks
\begin{tasklist} 
\begin{task}[id=foo,wphases=0-3,requires=\taskin{t1}{dissem}]
  To perform the administrative, scientific/technical, and financial management of the
  project 
\end{task}
\begin{task}[wphases=13-17!.5]
  To co-ordinate the contacts with the DFG and other funding bodies, building on the
  results in \taskref{management}{foo}
\end{task}
\begin{task}
  To control quality and timing of project results and to resolve conflicts
\end{task}
\begin{task}
  To set up inter-project communication rules and mechanisms
\end{task}
\end{tasklist}

\end{workpackage}
 
\begin{workpackage}[id=dissem,lead=pcg,
 title=Dissemination and Exploitation,
pcgRM=8,jacuRAM=2] 
Much of the activity of a project involves small groups of nodes in joint work. This work
 package is set up to ensure their best wide-scale integration, communication, and
 synergetic presentation of the results. Clearly identified means of dissemination of
 work-in-progress as well as final results will serve the effectiveness of work within the
 project and steadily improve the visibility and usage of the emerging semantic services.


 The work package members set up events for dissemination of the research and
 work-in-progress results for researchers (workshops and summer schools), and for industry
 (trade fairs). An in-depth evaluation will be undertaken of the response of test-users.
 
 \begin{tasklist}
  \begin{task}[id=t1,wphases=6-7]
    sdfkj
  \end{task}
  \begin{task}[wphases=12-13]
    sdflkjsdf
  \end{task}
  \begin{task}[wphases=18-19]
    sdflkjsdf
  \end{task}
 \begin{task}[wphases=22-24] 
 \end{task}
\end{tasklist}

Within two months of the start of the project, a project website will go live. This
website will have two areas: a members' area and a public area.\ldots
\end{workpackage}
\end{workarea}
 

\begin{workarea}[id=systems,title={System Development}]
  This workarea does not correspond to \OBJtref{secondobj}, but it has two work packages:
  one for the development of the {\LaTeX} class ({\WPref{class}}), and for the
  proposal template ({\WPref{temple}})

  This work group coordinates the system development.

\begin{workpackage}[id=class,lead=jacu,
                    title=A LaTeX class for EU Proposals,short=Class,
                   jacuRM=12,jacuRAM=8,pcgRM=12,pcgRAM=2]
We plan to develop a {\LaTeX} class for marking up EU Proposals

We will follow strict software design principles, first comes a
requirements analys, then \ldots
\begin{tasklist}
  \begin{task}[wphases=0-2]
    sdfsdf
  \end{task}
  \begin{task}[wphases=4-8]
    sdfsdf
  \end{task}
  \begin{task}[id=t3,wphases=10-14]
    sdfsdf
  \end{task}
  \begin{task}[wphases=20-24]
    sdfsdfd
  \end{task}
\end{tasklist}
\end{workpackage} 

\begin{workpackage}[id=temple,lead=pcg,
  title= Proposal Template,short=Template,jacuRM=12]

We plan to develop a template file for {\pn} proposals

We abstract an example from existing proposals
\begin{tasklist}
  \begin{task}[wphases=6-12]
    sdfdsf 
  \end{task}
  \begin{task}[id=temple2,wphases=18-24,requires=\taskin{t3}{class}]
    sdfsdf
  \end{task} 
\end{tasklist}
\end{workpackage}

\begin{workpackage}[id=workphase,title=A work package without tasks,
  wphases=0-4!.5]
  
  And finally, a work package without tasks, so we can see the effect on the gantt chart
  in fig~\ref{fig:gantt}.
\end{workpackage}
\end{workarea}
\end{workplan} 

\ganttchart[draft,xscale=.45] 

\subsection{Data Handling \deu{(Umgang mit den im Projekt erzielten Forschungsdaten)}}

All project results will be published for at least x years at our archive at \url{http://example.org}.

\subsection{Other Information \deu{(Weitere Angaben)}} Not applicable.

\subsection{Explanations on the proposed investigations \deu{(Erläuterungen zu den vorgesehenen Untersuchungen)}} Not applicable.

\subsection{Information on scientific and financial involvement of international cooperation partners \deu{(Erläuterungen zur inhaltlichen und finanziellen Projektbeteiligung von Kooperationspartnerinnen und Kooperationspartnern im Ausland)}} Not applicable.


%%% Local Variables: 
%%% mode: LaTeX
%%% TeX-master: "proposal"
%%% End: 

% LocalWords:  workplan.tex wplist dfgcount wa mansubsus duratio ipower wpfig
% LocalWords:  ganttchart xscale workplan workarea pdataref dissem workpackage foo
% LocalWords:  tasklist taskin taskref sdfkj sdflkjsdf sdfsdf sdfsdfd sdfdsf pn
% LocalWords:  firstobj secondobj pdatacount WAref ednote OBJref pcgRM pcg
% LocalWords:  ldots OBJtref workphase


\section{Bibliography concerning the state of the art, the research objectives, and the
  work programme \deu{(Literaturverzeichnis zum Stand der Forschung, zu den Zielen und dem
    Arbeitsprogramm)}}

\begin{todo}{from the proposal template}
In this bibliography, list only the works you cite in your presentation of the state of the
art, the research objectives, and the work programme. This bibliography is not the list
of publications. Non-published works must be included with the proposal.
\end{todo}
\printbibliography[heading=empty]
% the following will not become part of the public proposal after all most of this is
% technical or confidential.
%\begin{private}
\svnInfo $Id: funds.tex 22679 2011-12-01 07:08:45Z kohlhase $
\svnKeyword $HeadURL: https://svn.kwarc.info/repos/kwarc/doc/macros/forCTAN/proposal/dfg/examples/proposal/funds.tex $
\section{Requested Modules/Funds \deu{(Beantragte Module/Mittel)}}

For each applicant, we apply for funding within the Basic Module.

\subsection{Funding for Staff \deu{(Personalbedarf)}}\label{sec:positions}
\subsubsection{Research Staff}

We apply for the following positions. All run over the entire duration of the proposed project.

\paragraph*{Non-doctoral staff}\ednote{compute amount in elan and copy here}

One doctoral researcher for 2 years at $100 \%$ for Michael Kohlhase.

One doctoral researcher for 2 years at $100 \%$ for Florian Rabe.

%\paragraph*{Postdoctoral staff}
%\ednote{postdoctoral researcher and comparable}

\paragraph*{Other research assistants}\ednote{students with BSc.}

One student with BSc. for 2 years at $100 \%$ for Michael Kohlhase.

One student with BSc. for 2 years at $100 \%$ for Florian Rabe.

\subsubsection{Non-academic Staff} None.

\subsubsection{Student assistants} None.

\subsection{Funding for direct project costs}

\subsubsection{Equipment up to 10,000 \texteuro, software and consumables}

None.  PC will cover the workspace, computing needs, and consumables for its staff as part
of the basic support.

\subsubsection{Travel Expenses\deu{(Reisen)}}\label{sec:travel}

\begin{oldpart}{rework}
  The travel budget shall cover:
  \begin{itemize}
  \item visits to external collaborators. We expect two international visits. We estimate
    that each visit will be most effective, if the junior researchers can spend about 3
    weeks with the partners. Thus we estimate 2500 {\texteuro} per visit.
  \item visits to national conferences to disseminate the results of {\pn}. We expect
    one visit for each year for each of the three researchers. (3 x 3 x 1000 {\texteuro})
  \item visits to international conferences to disseminate the results of {\pn}. These
    are in particular the International Joint Conference on Document Engineering (DocEng)
    and the Tech User Group Meeting (TUG). We expect one visit for each proposed
    researcher and for each year. (3 x 3 x 1500 {\texteuro})
  \end{itemize}

  This sums up to a total amount of 32.500 {\texteuro} for travel expenses for the whole
  funding period of three years which is split into 16.250 {\texteuro} for each institute
  (PC and Jacobs University).
\end{oldpart}

\subsubsection{Visiting Researchers}

Total expenses \textbf{10.200 \texteuro}
\medskip

As explained in Section~\ref{sec:travel}, we expect 5 incoming research visits.  Assuming
an average duration of 3 weeks, we estimate the cost of one visit at 600 {\texteuro} for
traveling and 70 {\texteuro} per night for accommodation, amounting to 2040 \texteuro per
visit.

\subsubsection{Expenses for laboratory animals} None.

\subsubsection{Other costs \deu{(Sonstige Kosten)}} None.

\subsubsection{Project-related publication expenses} None.

\subsection{Funding for Instrumentation} None.

%%% Local Variables: 
%%% mode: LaTeX
%%% TeX-master: "proposal"
%%% End: 

% LocalWords:  ipower texteuro

\svnInfo $Id: preconditions.tex 22679 2011-12-01 07:08:45Z kohlhase $
\svnKeyword $HeadURL: https://svn.kwarc.info/repos/kwarc/doc/macros/forCTAN/proposal/dfg/examples/proposal/preconditions.tex $
\section{Project Requirements \deu{(Voraussetzungen f\"ur die Durchf\"uhrung des Vorhabens)}}

\subsection{Employment status information \deu{(Angaben zur Dienststellung)}}

\begin{todo}{from the proposal template}
For each applicant, state the last name, first name, and employment status (including
duration of contract and funding body, if on a fixed-term contract).
\end{todo}

\subsection{First-time proposal data \deu{(Angaben zur Erstantragstellung)}}

\begin{todo}{from the proposal template}
Only if applicable: Last name, first name of first-time applicant.

If this is your first proposal, reviewers will consider this fact when assessing your pro-
posal. Previous proposals for research fellowships, publication funding, travel allow-
ances, or funding for scientific networks are not considered first proposals. If you are
submitting a “first-time proposal” and it is part of a joint proposal, please note that your
independent project must be distinct from the other projects.

If you have already submitted a proposal as an applicant for a research grant and have
received a letter informing you of the funding decision, or if you have led an independ-
ent junior research group or project in a Collaborative Research Centre or Research
Unit, you are no longer eligible to submit a “first proposal”. If you have submitted a
“first-time proposal” and it was rejected, you may resubmit the application, in revised
form, as a first-time proposal for the same project.
\end{todo}

\subsection{Composition of the project group \deu{(Zusammensetzung der Projektarbeitsgruppe)}}

\begin{todo}{from the proposal template}
List only those individuals who will work on the project but will not be paid out of the
project funds. State each person’s name, academic title, employment status, and type
of funding.

Please list separately the individuals paid by your institution and those paid using other
third-party funding (including fellowships).
\end{todo}
\begin{sitedescription}{jacu}
The KWARC (Knowledge Adaptation and Reasoning for Content) research group headed by
Michael Kohlhase for has the following members
\begin{compactdesc}
\item[Dr. N.N.] is the \ldots She has a background in\ldots.
\end{compactdesc}
Additionally, the group has attracted about 10 undergraduate and master's students that
actively take part in the project work and various aspects of research.
\end{sitedescription}
\begin{sitedescription}{pcg}
Power Consulting GmbH is the leading provider of semantic document solutions. Dr. Senior
Researcher leads an applied research group consisting of 
\begin{compactdesc}
\item[Dr. N.N.] is the \ldots She has a background in\ldots.
\end{compactdesc}
The group has access to seven programming slaves specializing in web development and
document transformation techniques
\end{sitedescription}


\subsection{Cooperation with other researchers \deu{(Zusammenarbeit mit anderen Wissenschaftlerinnen und Wissenschaftlern)}}

\subsubsection{Researchers with whom you have agreed to cooperate on this project \deu{(Wissenschaftlerinnen und Wissenschaftler, mit denen für dieses Vorhaben eine konkrete Vereinbarung zur Zusammenarbeit besteht)}}

\begin{compactdesc}
\item[Prof. Dr. Super Akquisiteur (Uni Paderborn)] knows exactly what to do to get funding
  with DFG, we will interview him closely and integrate all his intuitions into the
  {\pn} templates.
\item[Prof. Dr. Habe Nichts (Uni Hinterpfuiteufel)] has never gotten a grant proposal
  through with DFG, we will try to avoid his mistakes.
\item[Dr. Sach Bearbeiter (DFG)] will consult with the DFG requirements to be met in the
  proposals.
\item[Dr. Donald Knuth (Stanford University)] is so surprised that we want to do grant
  proposals in {\TeX/\LaTeX} that he will help us with any problems we have in coding in
  this wonderful programming language.
\end{compactdesc}

\subsubsection{Researchers with whom you have collaborated scientifically within the past three years \deu{(Wissenschaftlerinnen und Wissenschaftler, mit denen in den letzten drei Jahren wissenschaftlich zusammengearbeitet wurde)}}

\ednote{Anmerkung Jens: Etwas unklar, was die DFG hier möchte. Die Liste der Personen kann
  sehr lang sein, also ist es wahrscheinlich besser nur die wichtigsten Projekte und
  Kontakte zu listen.}

\begin{todo}{from the proposal template}
This information will assist the DFG’s Head Office in avoiding potential conflicts of in-
terest during the review process.
\end{todo}


\subsection{Scientific equipment \deu{(Apparative Ausstattung)}}

Jacobs University provides laptops or desktop workstations for all academic
employees. Great Consulting GmbH. is rolling in money anyways and has all of the latest
gadgets.


\subsection{Project-relevant interests in commercial enterprises \deu{(Projektrelevante Beteiligungen an erwerbswirtschaftlichen Unternehmen)}}

Not applicable.

%%% Local Variables: 
%%% mode: LaTeX
%%% TeX-master: "proposal"
%%% End: 

% LocalWords:  Durchf uhrung subsubsection ipower Hinterpfuiteufel Sach Aktivit
% LocalWords:  Erkl arungen


\section{Additional information \deu{(Ergänzende Erklärungen)}}

Funding proposal XYZ-83282 has been submitted prior to this proposal on related topic XYZ.
\end{proposal}

\end{document}
 
%%% Local Variables: 
%%% mode: LaTeX
%%% TeX-PDF-mode:t
%%% TeX-master: t
%%% End: 

% LocalWords:  empty bibflorian systems rabe institutions modal historical pub
% LocalWords:  kwarc till formalsafe miko gc ipower ipowerlong Antr agen Beitr

% LocalWords:  acrolong intellegible kollaboratives koh arenten ussen Proze pcg
% LocalWords:  Versionsmanagementsystem textsc unterst utzt konzentieren stex
% LocalWords:  mechanik workplan thispagestyle newpage Principcal cvpubsmiko pn
% LocalWords:  ourpubs zusammenfassung printbibliography pubspage ntelligent
% LocalWords:  iting pnlong