%!TEX root = thesis.tex
\chapter{Introduction}
\label{chapter:Introduction}

\section{Overview}
\label{overview}

% [edited]
Understanding the phenomena of quantum many-body systems is one of the most challenging problems in condensed matter physics due to the coefficients required for describing entire systems grows exponentially with system size. For instance, If we desired to fully describe a $N$-site spin chain and each spin has $d$ probable states, the requirement of coefficients is $d^N$. As a result, it is impossible to simulate the system whose size is larger than 50 for classical computers due to the rapid increment of computational consumptions.

%[edited]
Various numerical methods have been developed to address the problems mentioned above. For instance, density matrix renormalization group (DMRG) \cite{PhysRevLett.69.2863} \cite{PhysRevB.48.10345} acquire the accuracy ground state energy and provide a dominant tool to study properties of one dimensional systems. Furthermore, the theory of DMRG is related with matrix product states (MPS) \cite{PhysRevB.73.094423} \cite{PhysRevLett.75.3537}, which could describe the wave-function of one dimensional system and be explicitly represented by \textit{tensor diagrams}. Therefore, DMRG is recognized as the one of the most successful method in one-dimensional systems. However, it's failure in higher dimensional systems due to the insufficiency of dealing with the entanglement in systems from matrix product states. For two-dimensional lattices, the projected entangled pair state (PEPS) \cite{PhysRevA.75.033605} \cite{jordan_studies_2011} has been applied to deal with that problem and many algorithms sprang up like mushrooms, such as the multiscale entanglement renormalization ansatz \cite{PhysRevLett.99.220405}.

On the other hand, the time-evolving block decimation (TEBD)\cite{PhysRevLett.91.147902} \cite{PhysRevLett.93.040502} provides another way to simulate with time evolution. According to the theory TEBD, Guifr\'e Vidal expend it to compute the ground state energy of an infinite systems (iTEBD) \cite{PhysRevLett.98.070201} \cite{PhysRevB.78.155117}. Due to the translational invariance taken into consideration, the consumption of the algorithms becomes independent of the size of the system. Despite it success in one-dimensional systems, the attempt to implement it in two-dimensional systems with the PEPS structure encounters in some problems. Firstly, the consumption enlarge with $D^4$, where $D$ is the dimension of the virtual state. Secondly, the environment strongly influences the accuracy and the stability of the algorithm. Therefor, many algorithms are developed to optimize the projected states and the measurement of the local observables, such as the projected entangled simplex state (PESS) \cite{PhysRevX.4.011025}, the tensor renormalization group (TRG) \cite{PhysRevLett.99.120601} \cite{PhysRevB.78.205116} \cite{PhysRevB.80.155131}, the high-order tree tensor network HOTRG \cite{PhysRevB.86.045139} and corner transfer matrix (CTM) \cite{doi:10.1143/JPSJ.65.891} \cite{PhysRevB.80.094403}.
