%!TEX root = thesis.tex
\chapter{Introduction}
\label{chapter:Introduction}

\section{Overview}
\label{overview}

% [edited]
Understanding the phenomena of quantum many-body systems is one of the most challenging problems in condensed matter physics because the parameters required to describe the entire systems grows with system size. For instance, if we desire to fully describe a $N$-site spin chain on which each spin has $d$ probable states, the required coefficients is $d^N$. As a result, it is impossible to solve exactly with computer for the large system size.

%[edited]
Therefore, various numerical methods have been developed to address this problems. For instance, the density matrix renormalization group (DMRG) \cite{PhysRevLett.69.2863, PhysRevB.48.10345} is a powerful variational method to obtain the ground state of one-dimensional systems. Based on the matrix product state (MPS) \cite{PhysRevB.73.094423, PhysRevLett.75.3537}, Vidal proposed another method, time-evolving block decimation (TEBD) to simulate time evolution of 1D systems. The TEBD \cite{PhysRevLett.91.147902, PhysRevLett.93.040502} method also allows the simulation of imaginary time evolution and can be generalized to simulate an infinite lattice \cite{PhysRevLett.98.070201, PhysRevB.78.155117}. 
%, which could describe the wave-function of one-dimensional systems and be explicitly represented by \textit{tensor diagrams}.
Despite their success in one dimensional systems, the attempt to extend to higher-dimensional systems encountered serious problems. Through many tests, we know in two dimensional systems, DMRG has difficulty dealing with long range correlation and hardly manage the entanglement. Methods  on projected entangled pair states (PEPS) \cite{PhysRevA.75.033605, jordan_studies_2011} ansatz are proposed to fulfill the entanglement area law in higher dimensions \cite{RevModPhys.82.277} \cite{} inaccurate but also inefficient due to the influence of environments and the growth of computational consumption. Therefore there are algorithms developed to optimize the wave function  and the measurement of the local observables, such as the projected entangled simplex state (PESS) \cite{PhysRevX.4.011025}, the tensor renormalization group \cite{PhysRevLett.99.120601,PhysRevB.78.205116,PhysRevB.80.155131}, the high-order tree tensor network HOTRG \cite{PhysRevB.86.045139} and corner transfer matrix (CTM) \cite{doi:10.1143/JPSJ.65.891,PhysRevB.80.094403}.

The thesis is organized as follow. Firstly, in Chapter.~\ref{chap2} we introduce the basic theory of the tensor network as a new language to describe many-body states. Next, in Chapter~.\ref{chapter:2ditebd} and ~\ref{chapter:ipess} we will introduce further optimization of two-dimensional iTEBD method compare the results obtained using PESS ansatz. Then, in Chapter.~\ref{chapter:ctm} we consider the environments in two dimensional systems and simulate the effective environments by CTM. Finally, the pros and cons of those algorithms are summarized in Chapter.~\ref{chapter:summary}.

