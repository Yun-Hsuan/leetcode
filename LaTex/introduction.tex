%!TEX root = thesis.tex
\chapter{Introduction}
\label{chapter:Introduction}

\section{Overview}
\label{overview}

% [edited]
Understanding the phenomena of quantum many-body systems is one of the most challenging problems in condensed matter physics due to the parameters required for describing entire systems grows exponentially with system size. For instance, if we desired to fully describe a $N$-site spin chain on which spin has $d$ probable states, the requirement of coefficients is $d^N$. As a result, it is impossible to solve exactly with computer for the large system size.

%[edited]
Therefore, various numerical methods have been developed to address the problems mentioned above. For instance, density matrix renormalization group (DMRG) \cite{PhysRevLett.69.2863, PhysRevB.48.10345}, which is a powerful variational method, acquire the accurate approximation of the ground state energy in one-dimensional systems. On the other hand, base on the theory of MPS \cite{PhysRevB.73.094423, PhysRevLett.75.3537}, Guifr\'e Vidal obtained another method, time evolving block decimation to compute the ground state. The TEBD \cite{PhysRevLett.91.147902, PhysRevLett.93.040502} method allows the simulation of time evolution and was simply generalized to simulate an infinite lattice (iTEBD) \cite{PhysRevLett.98.070201, PhysRevB.78.155117}. 
%, which could describe the wave-function of one-dimensional systems and be explicitly represented by \textit{tensor diagrams}.
Despite their success in one-dimensional systems, the attempt to extend them to higher-dimensional systems encountered in some problems. Through many tests, we have know that in two dimensional systems, DMRG is struggled to deal with the correlation and hardly manage the entanglement, and TEBD methods which are based on the projected entangled pair states (PEPS) \cite{PhysRevA.75.033605, jordan_studies_2011} ansatz is not only inaccurate but also inefficient due to the influence of environments and the growth of computational consumption. Therefore there are many algorithms were developed to optimize the projected states and the measurement of the local observables, such as the projected entangled simplex state (PESS) \cite{PhysRevX.4.011025}, the tensor renormalization group \cite{PhysRevLett.99.120601,PhysRevB.78.205116,PhysRevB.80.155131}, the high-order tree tensor network HOTRG \cite{PhysRevB.86.045139} and corner transfer matrix (CTM) \cite{doi:10.1143/JPSJ.65.891,PhysRevB.80.094403}.

The thesis is organized as follow. Firstly, in Chapter~.\ref{chap2} we introduce the basic theory of the tensor network which is a convenient language to describe many-body systems. Next, in Chapter~.\ref{chapter:2ditebd} and ~\ref{chapter:ipess} we will introduce some optimization of two-dimensional iTEBD and a new ansatz, projected entangled simplex state, to obtain and calculate the ground state more accurate. Then, in Chapter.~\ref{chapter:ctm} we consider the environments in two dimensional systems and simulate the effective environments by CTM. Finally, the pros and cons of those algorithms are summarized in Chapter.~\ref{chapter:summary}.

