%!TEX root = thesis.tex
\chapter{Infinite Projected Entangled Simplex State}
\label{chapter:ipess}
%%%%%%%%%%%%%%%%%
In this chapter, we introduce a new class of tensor network states called projected entangled simplex solid state (PESS). In Sec.~\ref{solidstate}, we briefly review the theory of simplex solid state \cite{PhysRevB.77.104404}. Then, based on that theory, we introduce how to implement PESS algorithms and apply it to study kagome and square systems in Sec.~\ref{vpess}, Sec.~\ref{kagome} and Sec.~\ref{4pess2b}. Finally, we show the results to illustrate the properties of PESS ansatz in Sec.~\ref{ppess} .
%%%%%%%%%%%%%%%%%
\section{Simplex-solid State}
\label{solidstate}
The simplex-solid state of an SU(N) quantum anti-ferromagnet was first derived by Arovas \cite{PhysRevB.77.104404} and the most significant property is that any simplex states could be described by a natural generalization of the AKLT \cite{PhysRevLett.59.799} \cite{Affleck1988} valence bond solid state. It means that the bond signlets of the AKLT could be extended to $N$-site simplices.

The concept of simplex-solid states ansatz were introduce by Xie et al \cite{PhysRevX.4.011025, PhysRevB.93.075154}. The tensor-network representation of simplex states is called projected entangled simplex state (PESS) which is also considered as an extension of PEPS. It obeys the area law \cite{RevModPhys.82.277} of entanglement entropy and is able to characterize two-dimensional systems if the dimension of the virtual bonds are large enough. Unlke PEPS, the entanglement among the virtual particles is described by entangled simplex tensors which depends on the structure of PESS. For example, in Fig.~\ref{fig411}(b), on a kagome lattice there are three virtual particles within the simplex state so the entangled simplex tensor $S_{mnl}$ is a rank-3 tensor.

\begin{figure}[ht]
	\centering
	\includegraphics[width=0.80\textwidth]{figures/fig411.png}
	\caption[The simplex solid state on the kagome lattice.]{The simplex solid state on the kagome lattice.}
	\label{fig411}
\end{figure}

To illustrate how to write down the formulation of the PESS wave function, we begin from a simple system, the spin-2 simplex state on an infinite Kagome lattice. In Fig.~\ref{fig411}(a), each physical $S=2$ states on the lattices could be treated as a symmetric superposition of two virtual $S=1$ spins. According to the theory of AKLT, each neighbor simplices (triangles) share a single site symmetrically, which also means that the $S=1$ spins could be assigned to one of the simplices (vertex-sharing). Hence, there are three $S=1$ spins in each simplex state. From the features of the $SU(2)$ group, the decomposition of the direct-product of three integer spins is written as, 
\begin{align}
	\label{su2}
	n \otimes n \otimes n = [a_0 \times 0] \oplus \dots \oplus [a_k \times k] \oplus \dots \oplus [a_{3n} \times 3n] \\
	a_k = \begin{cases}
		2k + 1 & \text{, $k \leq n$} \\
		3n + 1 - k & \text{, $k > n$} 
	\end{cases}
	\quad \text{, } k = 0, 1, \dots , 3n ,
\end{align}
where $a_k$ is a constant and $k$ indicates the $k$th irreducible representation. Now that we can write down the product of the spins in the simplex as,  
\begin{align}
	\label{111}
	\barbelow{1} \otimes \barbelow{1} \otimes \barbelow{1} = \barbelow{0} \oplus \left( 3 \times \barbelow{1} \right) \oplus \left( 2 \times \barbelow{2} \right) \oplus \barbelow{3},
\end{align}
and show that there is an unique spin-singlet state. The result encourages us to define a virtual singlet on simplex,
\begin{align}
	\Ket{\psi_{\alpha}} = \frac{1}{\sqrt{6}} \sum_{\left\{ s_i s_j s_k\right\}}{S^{\alpha}_{s_i s_j s_k}\Ket{s_i, s_j, s_k}},
\end{align}
where $s_i,s_j,s_k$ are $S=1$ virtual spins located at site $i,j$ and $k$ of the simplex $\alpha$ and $S^{\alpha}_{s_i s_j s_k}$ is the Levi-Civita antisymmetric tensor $\varepsilon_{ijk}$ [Fig.~\ref{fig411}(b)]. In order to map the two virtual $S=1$ spins to the spin-2 subspace, we define the projection operator $P_i$ $P_j$ and $P_k$ on each site,
\begin{align}
	&P_i = \sum_{s_i,s^{\prime}_{i}}\sum_{\sigma_i}{A^{\sigma_i}_{s_i,s_i^{\prime}}\Ket{\sigma_i}\Bra{s_i,s_i^{\prime}}} \\
	&P_j = \sum_{s_j,s^{\prime}_{j}}\sum_{\sigma_j}{A^{\sigma_j}_{s_j,s_j^{\prime}}\Ket{\sigma_j}\Bra{s_j,s_j^{\prime}}} \\
	&P_k = \sum_{s_k,s^{\prime}_{k}}\sum_{\sigma_k}{A^{\sigma_k}_{s_k,s_k^{\prime}}\Ket{\sigma_k}\Bra{s_k,s_k^{\prime}}}
\end{align}
where $\Ket{\sigma_i}$, $\Ket{\sigma_j}$ and $\Ket{\sigma_k}$ are the basis of the $S=2$ spins at site $i$, $j$ and $k$. $A^{\sigma_i}_{s_i,s_i^{\prime}}$ is a projected matrix whose components are filled by the Clebsch-Gordan coefficients,
\[
	\begin{aligned}
		&A_{11}^2 = A_{33}^{-2} = 1, \\
		&A_{12}^1 = A_{21}^{1} = A_{23}^{-1} = A_{32}^{-1} = \frac{1}{\sqrt{2}}, \\
		&A_{13}^0 = A_{31}^{0} = \frac{1}{\sqrt{6}}, \\
		&A_{22}^0 = \frac{2}{\sqrt{6}},
	\end{aligned}
\]
Now that we can write down the wave function of the simplex-solid state,
\begin{align}
	\Ket{\Phi} &= \bigoplus_i P_i \prod_{\alpha}{\Ket{\psi_{\alpha}}} \\
	&= \Tr \left( \dots S_{s_i s_j s_k}^{\alpha} A^{\sigma_i}_{s_i,s_i^{\prime}} A^{\sigma_j}_{s_j,s_j^{\prime}} A^{\sigma_k}_{s_k,s_k^{\prime}} \dots \right) \Ket{\dots \sigma_i \sigma_j \sigma_k \dots}.
\end{align}
and apply the tensor-network representation to describe it, as shown in Fig.~\ref{fig411}(b).
This structure could be extended to any higher integer spins. In conlusion, a physical $S=2n$ (even-integer) spin is considered as a symmetric superposition of two virtual $S=n$ spins and a $S=2n-1$ (odd-integer) one is regarded as a symmetric superposition of a virtual $S=n-1$ spin and a virtual $S=n$ spin [ Fig.~\ref{fig411}(a) ]. More details are presented in reference \cite{PhysRevB.93.075154} \cite{0953-8984-21-45-456009}.

\section{Variational PESS ansatz}
\label{vpess}
In this section, we will employ the formulation of the PESS wave function as a variational ansatz. The PESS ansatz is similar to the imaginary time evolution discussed in chapter.\ref{chapter:2ditebd}. However, unlike the PEPS algorithms, we apply a higher rank tensor $S$ to describe the entanglement entropy among virtual particles in a simplex state. Due to the difference, we use \textit{high-order singular value decomposition} (HOSVD) \cite{PhysRevB.86.045139} \cite{doi:10.1137/S0895479896305696} rather than SVD to decompose the wave function.
\subsection{High-order singular value decomposition}
\label{hosvd}
In this section, we will introduce the $N$th-order singular value decomposition or so called high-order SVD (HOSVD), which is proposed for decomposing rank-$N$ tensors, and show the pseudo-code to illustrate the scheme of the implementation.
\begin{figure}[b]
	\centering
	\includegraphics[width=1.00\textwidth]{figures/fig4311.png}
	\caption[The picture of HOSVD for a rank-3 tensor.]{The picture of HOSVD for a rank-3 tensor $A$. $U^{(1)}$,$U^{(2)}$ and $U^{(3)}$ are unitary matrices and $S$ (yellow cuboid) is a core tensor.}
	\label{fig4311}
\end{figure}

According to the theorem of HOSVD \cite{doi:10.1137/S0895479896305696}, every \textit{complex} ($ I_1 \times I_2 \times \dots \times I_N$)-tensor $A$ could be decomposed to a \textit{core tensor} $S$ and other $n$-mode unitary matrix $U^{(n)}$, where $n$ must be smaller than $N$,
\begin{align}
	A = S \times U^{(1)} \times U^{(2)} \times \dots \times U^{(n)}
\end{align} 
As shown in Fig.~\ref{fig4311}, a rank-3 tensor $A$ is decomposed into a core tensor $S$ (yellow cuboid) and three unitary matrices obtaining from three different modes.  The core tensor $S$ is a \textit{complex} rank-$n$ tensor and there are two significant properties, \begin{enumerate}
	\item Row-orthogonal: Two subtensors $S_{\alpha}^{(n)}$ and $S_{\beta}^{(n)}$ are orthogonal when $\alpha \neq \beta$. 
		\begin{align}		
			\left\langle S^{(n)}_{\alpha},S^{(n)}_{\beta} \right\rangle \quad \text{, if} \quad \alpha = \beta
		\end{align}		
		In other words, no matter what the shape of $S$ is, any two rows are orthogonal.
	\item Ordering: The Frobenius-norms $\text{ } \norm{S^{(n)}_{i}} $ is ordered from large to small.
		\begin{align}		
			\norm{S^{(n)}_1} \geq \norm{S^{(n)}_2} \geq \dots \geq \norm{S^{(n)}_{I_n}} \geq 0,
		\end{align}		
		for all possible values of $n$.
\end{enumerate}
\begin{figure}[t]
	\centering
	\includegraphics[width=0.90\textwidth]{figures/fig4312.png}
	\caption[The tensor-network representation of HOSVD]{(a) Decompose $A$ to a core tensor $S$ and $n$-mode unitary matrix $U^{(n)}$, (b) Obtain tensors $U^{(n)}$ from decomposing various modes of tensor $A$ by SVD, (c) Obtain the core tensor $S$ from contracting all transpose unitary tensors $U^{(n)T}$ }
	\label{fig4312}
\end{figure}
For instance, if the goal is to decompose a rank-3 $(N=3)$ tensor $A$ from 3 $(n=3)$ different modes, as in Fig.~\ref{fig4312}(a), we can implement it by the following steps, 
\begin{enumerate}
	\item	Reshaping the tensors to each modes and obtain $U^{(n)}$ by SVD , as shown in Fig.~\ref{fig4312}(b).
	\item	Contract all $U^{(n)T}$ tensors for calculating $S$, 
		\begin{align}
			S = A \times U^{(1)T} \times U^{(2)T} \times \dots \times U^{(n)T}
		\end{align} 
		see Fig.~\ref{fig4312}(c).
\end{enumerate}
\subsection{Simple update for PESS}
In Chap.~\ref{chapter:2ditebd}, we have introduced an efficient approximation, the "simple update", for determining the ground state in the PEPS. In principle, the wave-function of PESS can be approached by the same way. In PEPS structure, the ground-state wave function is approximated by iterated application of imaginary-time evolution operators $U(\tau) = e^{-\tau H}$ on an random PEPS state $\Ket{\Psi_t}$, where $\tau$ is a small constant, see Fig.~\ref{fig312}. We can obtain the ground-state wave function of PESS by following steps, which is similar to the simple update of PEPS
\begin{enumerate}
	\item Split the Hamiltonian $H$,  
		\begin{align}
			\label{splitH}
			H = H_{\alpha} + H_{\beta}
		\end{align}
		where $\alpha$ and $\beta$ are dependent on the geometry of the PESS structure.
	\item Obtain the evolution operator $U(\tau)$ by Trotter-Suzuki decomposition.
		\begin{align}
			e^{-\tau H} = e^{-\tau H_{\alpha}} e^{-\tau H_{\beta}} + O(\tau^{2}).
		\end{align}
	\item Absorb the environment bond vectors, which are considered as the entanglement between each simplex state, into projection tensors, and contract all projection tensors in the simplex state with the core tensor to obtain a cluster tensor.
	\item Apply the evolution operator $U$ the cluster tensor for obtaining a new cluster tensor.
	\item Decompose the new cluster tensor by HOSVD.
	\item Truncate all the projection tensors and the core tensor.
	\item Absorb the inverse bond vectors for removing the entanglement on the projection tensors.
\end{enumerate}
which is similar to simple update of PEPS.

\section{Infinite Kagome lattice}
\label{kagome}
In the previous section, we have briefly introduce to the procedures of PESS algorithm. The first step is to  choose an suitable $n$-PESS structure, where $n$ is the number of projection tensors in a simplex, to split the Hamiltonian and describe the system. There are many various graphical representations for the Kagome lattice [Fig.~\ref{fig411}(a)], such as 3-PESS, 5-PESS and 9-PESS \cite{PhysRevX.4.011025}. In this section, we will discuss more details about 3-PESS and 5-PESS.

\subsection{3-PESS}

In the 3-PESS case, the PESS state can be drawn as in Fig.~\ref{fig4321}(a). It shows pictures that the many-body states can be described by composing two different simplices, upward- and downward-oriented triangular as shown in Fig \ref{fig4321}(b). Each simplices contain three rank-3 projection tensors, $A^{\sigma_i}_{mm^{\prime}}$ $A^{\sigma_j}_{nn^{\prime}}$ and $A^{\sigma_k}_{ll^{\prime}}$ and a rank-3 core tensor, $S_{mnl}^{\alpha}$ or $S_{m^{\prime}n^{\prime}l^{\prime}}^{\beta}$. Therefore, the form of the Hamiltonian is rewritten by Eq.~\ref{splitH} as,
\begin{align}
	H = H_{\alpha} + H_{\beta} \quad,\text{where} \quad H_{\alpha} = H_{\Delta},\quad H_{\beta} = H_{\nabla}
\end{align}
Next, utilize the imaginary-time evolution operator $e^{-\tau H}$ to an random initial state $\Ket{\Psi_{0}}$ iteratively. According to the theorem of Trotter-Suzuki decomposition \cite{Suzuki01111976}, the evolution operator can be decomposed into two product terms, as $\tau \rightarrow 0$. Hence, the evolution operator operator of 3-PESS can be written as, 
\begin{align}
	\label{eq415}
	e^{-\tau H} = e^{-\tau H_{\Delta}} e^{-\tau H_{\nabla}} + O(\tau^2)
\end{align}
when the value of $\tau$ is small enough. 
\label{3pess}
\begin{figure}[ht]
	\centering
	\includegraphics[width=1.00\textwidth]{figures/fig4321.png}
	\caption[The graphical representation of 3-PESS]{The graphical representation of 3-PESS. (a) The PESS state can be considered as composed of two simplices, upward- (green triangle) and downward-triangular (green triangular). The red dash-line represents the geometry of the kagome lattice. As this figure shown, the projection tensors (Blue circle) are rank-3 with the dimension $dD^2$, where $d$ is the dimension of the physical basis and $D$ is the dimension of the virtual bonds (gray line), and the entangled simplex tensors, with the dimension $D^3$, are located at the cross of three virtual bonds (gray line) in each simplex states. (b) Two simplices in 3-PESS structure. These two type simplices, clustered with the sharing projection tensors, $A^{\sigma_i}_{mm^{\prime}}$, $A^{\sigma_j}_{nn^{\prime}}$ and $A^{\sigma_k}_{ll^{\prime}}$. However, their entangled simplex tensor are individual. In (a), the core tensor of the red simplex is $S^{\alpha}_{mnl}$ and the green one is $S^{\beta}_{m^{\prime}n^{\prime}l^{\prime}}$}
	\label{fig4321}
\end{figure}

After determining the evolution operators, $e^{\tau H_{\Delta}}$ and $e^{\tau H_{\nabla}}$, we apply the scheme of the simple update to approximate the wave-function of the ground state. As shown in Fig.~\ref{fig4323}, we take the upward-triangular simplex $\alpha$ as an example to illustrate the update scheme,

\begin{figure}[ht]
	\centering
	\includegraphics[width=1.00\textwidth]{figures/fig4322.png}
	\caption[The scheme of the simple update for 3-PESS.]{The scheme of the simple update for 3-PESS. The more detailed descriptions are in the paragraph}
	\label{fig4322}
\end{figure}

\begin{enumerate}
	\item Obtain a clustered simplex tensor $\Theta$: Firstly, absorb all the environment bond vectors surrounding the simplex state. See Fig.~\ref{fig4322}(a),  the environment vectors, $\lambda^{\beta}_{m^{\prime}m}$ , $\lambda^{\beta}_{n^{\prime}n}$ and $\lambda^{\beta}_{l^{\prime}l}$, are obtained from the entangled simplex tensor of simplex $\beta$, due to the canonical form of the PESS,
		\begin{align}
			&\sum_{n,l}{S^{(\beta)}_{mnl}\left( S^{(\beta)}_{m^{\prime}nl}\right)} = \delta_{m,m^{\prime}} \lambda^{(\beta)^2}_{m} \\
			&\sum_{l,m}{S^{(\beta)}_{mnl}\left( S^{(\beta)}_{mn^{\prime}l}\right)} = \delta_{n,n^{\prime}} \lambda^{(\beta)^2}_{n} \\
			&\sum_{m,n}{S^{(\beta)}_{mnl}\left( S^{(\beta)}_{mnl^{\prime}}\right)} = \delta_{l,l^{\prime}} \lambda^{(\beta)^2}_{l}
		\end{align}
		and based on the features of HOSVD. The $\lambda$s can be simply obtained, when we decompose the clustered tensors of the simplex $\beta$ in the updating process. Secondly, contract all tensors in simplex state to obtain the clustered tensor $\Theta$. The mathematical representation of the step is written as,  
		\begin{align}
			\Theta_{m^{\prime} n^{\prime} l^{\prime}}^{\sigma_i \sigma_j \sigma_k} = \sum_{mnl,m^{\prime}n^{\prime}l^{\prime}}{S^{(\alpha)}_{mnl} \lambda^{(\beta)}_{m^{\prime \prime}} A^{\sigma_i}_{mm^{\prime}} \lambda^{(\beta)}_{n^{\prime \prime}} A^{\sigma_j}_{nn^{\prime}} \lambda^{(\beta)}_{l^{\prime \prime}} A^{\sigma_k}_{ll^{\prime}}}
		\end{align}
	\item Apply the evolution operator $U(\tau)$: we can generate the evolution operator for 3-PESS ansatz by the Eq.~\ref{eq415}. For updating upward-triangular simplex, $U(\tau)$ is defined as, 
		\begin{align}
			U(\tau) = e^{-\tau H_{\Delta}}
		\end{align}
		Now we can utilize $U(\tau)$ to the cluster tensor $\Theta$, as in Fig.~\ref{fig4322}(c) and obtain a new cluster tensor $\Theta^{\prime}$, 
	\item Decompose $\Theta^{\prime}$ into the general form of the simplex state: Apply the high-order singular value decomposition (HOSVD) to obtain new projection tensors, $U^{(\sigma_i)}$, $U^{(\sigma_j)}$ and $U^{(\sigma_k)}$, and a new core tensor of simplex $\alpha$, $S^{(\alpha)^{\prime}}$, see Fig.~\ref{fig4322}(d). During operating HOSVD, we need to save the environment bond vectors, $\lambda^{(\alpha)}_{m}$, $\lambda^{(\alpha)}_{n}$ and $\lambda^{(\alpha)}_{l}$, surrounding the simplex $\beta$. All $\lambda^{(\alpha)}$ could be obtained from decomposing different modes of $\Theta^{\prime}$. As shown in Fig.~\ref{fig4312}(b).
	\item Truncation: In order to avoid the exponential increment of the virtual bonds dimension, we need truncate the brown bonds in Fig.~\ref{fig4322}(d) to specified dimension $D^{\prime}$ which can be fixed to original dimension $D$ or determined dynamically by setting an truncation error as the discussion in Sec.~\ref{truncaterr}. The simple way is to truncate projection tensors firstly,  
		\begin{align}
			&U^{\sigma_i} \rightarrow \widetilde{U}^{\sigma_i}_{mm^{\prime}} \\
			&U^{\sigma_j} \rightarrow \widetilde{U}^{\sigma_j}_{nn^{\prime}} \\
			&U^{\sigma_k} \rightarrow \widetilde{U}^{\sigma_k}_{ll^{\prime}}
		\end{align}
		Next, contract the transpose of these tensors with $\Theta^{\prime}$, as Fig.~\ref{fig4312}(c), to obtain $\widetilde{S}^{\alpha}$ in Fig.~\ref{fig4322}(e), where
		\begin{align}
			\widetilde{S}^{(\alpha)} = \sum_{m^{\prime}n^{\prime}l^{\prime}}{\Theta_{m^{\prime} n^{\prime} l^{\prime}}^{\prime \sigma_i \sigma_j \sigma_k} \widetilde{U}^{\sigma_i}_{mm^{\prime}} \widetilde{U}^{\sigma_j}_{nn^{\prime}} \widetilde{U}^{\sigma_k}_{ll^{\prime}}}
		\end{align}
	\item Absorb the inverse environment bond vectors $\lambda^{(\beta)^{-1}}$ into truncated projection tensors: To obtain the updated projection tensors $\widetilde{A}^{\sigma_i}_{mm^{\prime}}$, $\widetilde{A}^{\sigma_j}_{nn^{\prime}}$ and $\widetilde{A}^{\sigma_k}_{ll^{\prime}}$, we need to remove the influence of environment, 
		\begin{align}
			\widetilde{A}^{\sigma_i}_{mm^{\prime}} &= \sum_{m^{\prime \prime}}{ \lambda^{(\beta)}_{m^{\prime} m^{\prime \prime}} \widetilde{U}^{\sigma_i}_{mm^{\prime \prime}}} \\
			\widetilde{A}^{\sigma_j}_{nn^{\prime}} &= \sum_{n^{\prime \prime}}{ \lambda^{(\beta)}_{n^{\prime} n^{\prime \prime}} \widetilde{U}^{\sigma_j}_{nn^{\prime \prime}}} \\
			\widetilde{A}^{\sigma_k}_{ll^{\prime}} &= \sum_{l^{\prime \prime}}{ \lambda^{(\beta)}_{l^{\prime} l^{\prime \prime}} \widetilde{U}^{\sigma_i}_{ll^{\prime \prime}}}
		\end{align}	
		see Fig.~\ref{fig4322}(f),
\end{enumerate}

At here, one updated epoch is completed. Finally, the ground state will be obtained by iterating the procedures for each simplex, upward- and downward- triangular $(\Delta, \nabla)$, until the wave function of PESS converges, 
%$\mathlarger{\rjoin}$
\subsection{5-PESS}
\label{5pess}
In the 5-PESS structure [Fig.~\ref{fig4323}(a)], the procedure to approximate the ground state wave function is similar to 3-PESS. However, due to the transformation of the structure, the simplices should be re-defined. In the 5-PESS ansatz, each simplex state is composed of an upward- and a downward-triangular and the most significant difference is that the core tensors are located on the physical sites. As shown in Fig.~\ref{fig4323}(b), each simplex contains 4 projection tensors $A^{\sigma_i}_{ii^{\prime}}$ $A^{\sigma_j}_{jj^{\prime}}$ $A^{\sigma_k}_{kk^{\prime}}$ and $A^{\sigma_l}_{ll^{\prime}}$, and a core tensor, $S^{\sigma_s (\alpha)}_{ijkl}$ or $S^{\sigma_s (\beta)}_{i^{\prime}j^{\prime}k^{\prime}l^{\prime}}$. Therefore, we can split the Hamiltonian into 
\begin{align}
	H = H_{\alpha} + H_{\beta} \quad \text{,where} \quad H_{\alpha} = H_{\rjoinr}, H_{\beta} = H_{\rjoing}
\end{align}
and as in previous section, the evolution operator is written as, 
\begin{align}
	e^{-\tau H} = e^{-\tau H_{\rjoinr}} e^{-\tau H_{\rjoing}} + O(\tau^2)
\end{align}
when $\tau \rightarrow 0$.

\begin{figure}[ht]
	\centering
	\includegraphics[width=1.00\textwidth]{figures/fig4323.png}
	\caption[The graphical representation of 5-PESS]{The graphical representation of 5-PESS. (a) The PESS state can be considered as composed of two simplices, $\rjoinr$ and $\rjoing$. The red dash-line represents the geometry of the kagome lattice. As this figure shown, the projection tensors (Blue circle) are rank-3 with the dimension $dD^2$, where $d$ is the dimension of the physicla basis and $D$ is the dimension of the virtual bonds (gray line), and the entangled simplex tensors, with the dimension $dD^4$, are located at the center physical sites in simplices. (b) Two simplices in 5-PESS structure. These two type simplices, clustered with the sharing projection tensors, $A^{\sigma_i}_{ii^{\prime}}$, $A^{\sigma_j}_{jj^{\prime}}$, $A^{\sigma_k}_{kk^{\prime}}$ and $A^{\sigma_l}_{ll^{\prime}}$. However, their entangled simplex tensor are individual. In (a), the core tensor of the red simplex is $S^{\alpha}_{\sigma_s ijkl}$ and the green one is $S^{\beta}_{\sigma_s k^{\prime}j^{\prime}k^{\prime}l^{\prime}}$}
	\label{fig4323}
\end{figure}
Next, apply the simple update scheme as Fig.~\ref{fig4324}. Most of steps are similar to 3-PESS, shown as Fig.~\ref{fig4322}. However, we should carefully decompose the cluster tensor $\Theta^{\prime}$, where
\begin{align}
	\Theta^{\prime} = \Theta \times U(\tau)
\end{align}
see Fig.~\ref{fig4324}(d). In this structure, the core tensors contain physical basis states, which means that it can not be decomposed by the simply way shown in Fig.~\ref{fig4312} due to
\begin{align}
	N \mod n \neq 0
\end{align}
where $N$ is the rank of the tensor $\Theta^{\prime}$ and $n$ is the mode number of HOSVD. Hence, the bond $\sigma_s$, belonging to the core tensor $S^{(\alpha)}_{i^{\prime}j^{\prime}k^{\prime}l^{\prime}}$ in tensor $\Theta^{\prime}$, must be fixed during the decomposition. More details and general form of the HOSVD are written in the documentation of \textit{Uni10 Library} \cite{1742-6596-640-1-012040}, which is implemented by c/c++ and helpful for calculating the high-rank linear algebra problems.
\begin{figure}[ht]
	\centering
	\includegraphics[width=1.00\textwidth]{figures/fig4324.png}
	\caption[The scheme of the simple update for 5-PESS.]{The scheme of the simple update for 5-PESS. The more detailed descriptions are in the paragraph}
	\label{fig4324}
\end{figure}
\section{Infinite Square lattice}
\label{4pess2b}
In order to test and recognize the properties of the PESS ansatz more explicitly, we extend it to simulate infinite square systems and compare it with methods discussed in the Chap.~\ref{chapter:2ditebd}.

\subsection{4-PESS (Rank-3 projection tensors)}
The 4-PESS structure could be built by two different methods, which are drawn in the Ref.~\cite{PhysRevX.4.011025}. In this section, we use the concept, shown in Fig.~\ref{fig4325}, to complete the implementation. 

See Fig.~\ref{fig4325}(a), the many-body state can be regarded as a composite of two different simplices, as shown in Fig.~\ref{fig4325}(b), repeatedly. The components of each simplix are similar to 5-PESS, but the difference is that the core tensors, $S^{(\alpha_)}_{ijkl}$ and $S^{(\beta)}_{i^{\prime}j^{\prime}k^{\prime}l^{\prime}}$, are not located on the lattice sites. Hence, the Hamiltonian is split into,
\begin{align}
	H = H_{\alpha} + H_{\beta} \quad,\text{where} \quad H_{\alpha} = H_{\color{red} \square},\quad H_{\beta} = H_{\color{green} \square}
\end{align}
where ${\color{red} \square}$ and ${\color{green} \square}$ are represented simplex $\alpha$ and $\beta$, shown in Fig.~\ref{fig4325}(b). Therefore, the evolution operator can be written as, 
\begin{align}
	e^{-\tau H} = e^{-\tau H_{\color{red} \square}} e^{-\tau H_{ \color{green} \square}} + O(\tau^2)
\end{align}
when $\tau$ is a small constant $\rightarrow 0$.
\begin{figure}[ht]
	\centering
	\includegraphics[width=1.00\textwidth]{figures/fig4325.png}
	\caption[The graphical representation of 5-PESS]{The graphical representation of 5-PESS. (a) The PESS state can be considered as composed of two simplices, ${\color{red} \square}$ and ${\color{green} \square}$. The red dash-line represents the geometry of the kagome lattice. As this figure shown, the projection tensors (Blue circle) are rank-3 with the dimension $dD^2$, where $d$ is the dimension of the physicla basis and $D$ is the dimension of the virtual bonds (gray line), and the entangled simplex tensors, with the dimension $D^4$, are located at the cross of three virtual bonds (gray line) in each simplex states. (b) Two simplices in 3-PESS structure. These two type simplices, clustered with the sharing projection tensors, $A^{\sigma_i}_{ii^{\prime}}$, $A^{\sigma_j}_{jj^{\prime}}$, $A^{\sigma_k}_{kk^{\prime}}$ and $A^{\sigma_l}_{ll^{\prime}}$. However, their entangled simplex tensor are individual. In (a), the core tensor of the red simplex is $S^{\alpha}_{ijkl}$ and the green one is $S^{\beta}_{i^{\prime}j^{\prime}k^{\prime}l^{\prime}}$}
	\label{fig4325}
\end{figure}

Finally, the ground state wave function can be obtained by repeating the following steps, shown in Fig.~\ref{fig4326}, on the simplices, $\alpha$ and $\beta$ $( \color{red} \square$, ${\color{green} \square})$, until the wave function of PESS converge.

Before we do more restrict comparisons, it is obvious that the control of dimensional increment in the 4-PESS is better than in the PEPS [Fig.~\ref{fig314}], In 4-PESS structure, the projection tensors are described by rank-3 tensors whose dimension is $dD^2$. However, in the PEPS representation, the dimension of local tensors is $dD^4$. The reduction of tensors dimension is helpful for calculating and studying with a significant larger dimension. Nevertheless, it doesn't also mean that the 4-PESS algorithm is more efficient than PEPS one and actually it encounter in some problems when approximating the environment. We will show mare details in following sections.

\begin{figure}[ht]
	\centering
	\includegraphics[width=1.00\textwidth]{figures/fig4326.png}
	\caption[The scheme of the simple update for 4-PESS, composed by rank-3 projection tensors.]{The scheme of the simple update for 4-PESS, composed by rank-3 projection tensors. The more detail descriptions are in the paragraph}
	\label{fig4326}
\end{figure}

\section{Properties of PESS algorithm}
\label{ppess}

\subsection{3PESS on infinite Kagome lattice}

Before applying the PESS algorithms to study square lattice systems, we should know and verify some features. At beginning, we present the results for the Heisenberg model on Husimi lattice \cite{Husimi1, Husimi2},

\subsubsection{$S=1$}

\begin{figure}[!hb]
	\centering
	\includegraphics[width=0.7\textwidth]{figures/3pess_S1GE.png}
	\caption[The Energy per site of the $S=1$ Heisenberg model as a function of the virtual bond dimension $D$]{The Energy per site of the $S=1$ Heisenberg model as a function of the virtual bond dimension $D$.}
	\label{fig4327}
\end{figure}

In the case of $S=1$ Heisenberg model on the Husimi lattice, the ground state energy per sites $E_0$ decrease with virtual dimension $D$ and converge to $E_0=-1.405$ rapidly, see Fig.\ref{fig4327}. However, according to the theory of the simplex solid states [Sec. \ref{solidstate}], a physical $S=2n-1$ spin can be regarded as a symmetric superposition of a virtual $S=n-1$ and a virtual $S=n$ spin. We predict that there are two different ground state energy between the two simplices (upward- and downward-triangular). As shown in Fig.~\ref{fig4328}, the energy difference
\begin{align}
	\Delta E = 2 \frac{|E_{\Delta}-E_{\nabla}|}{3}
\end{align}
and the spontaneous magnetization
\begin{align}
	M = \frac{1}{N} \sum_{i}{\sqrt{\braket{S_x}^2 + \braket{S_y}^2 + \braket{S_z}^2}}
\end{align}
can be considered as two different order parameters. The energy difference $\Delta E$ rapid increase to a constant value and the magnetic order parameter $M$ begin to rapid fall to zero almost at the same time, where $D=8$. It means that to describe the simplices states, the requirement of virtual bond dimension $D_c$ must be larger than $8$. The results is also corresponds to the theory of the simplex solid states.

\begin{figure}[!t]
	\centering
	\includegraphics[width=0.75\textwidth]{figures/3pess_MDE.png}
	\caption[The trimerization parameter, $\Delta E = 2 |E_{\Delta}-E_{\nabla}| / 3$, and the local magnetization $M$ of the $S=1$ Heisenberg model as functions of the virtual bond dimension $D$]{The trimerization parameter, $\Delta E = 2 |E_{\Delta}-E_{\nabla}| / 3$, and the local magnetization $M$ of the $S=1$ Heisenberg model as functions of the virtual bond dimension $D$.}
	\label{fig4328}
\end{figure}

%\begin{figure}[H]
%	\centering
%	\includegraphics[width=0.80\textwidth]{figures/3pess_GEN.png}
%	\caption[The scheme of the simple update for 4-PESS, composed by rank-3 projection tensors.]{The scheme of the simple update for 4-PESS, composed by rank-3 projection tensors. The more detail descriptions are in the paragraph}
%	\label{fig4329}
%\end{figure}

\subsubsection{$S=2$}

Turning to the $S=2$ Heisenberg model, the ground state energy per site again converge to $E_0 = -4.8185$ [Fig.~\ref{fig4330}] when the dimension $D$ is sufficient enough. Unlike the $S=1$ case, each spins on sites can be considered as a symmetric superposition of two $S=1$ spins, which means that the ground state is a uniform simplex-solid state. Hence, theoretically there is no energy gap between two different simplices (upward- and downward-triangle). Then see Fig.~\ref{fig4331}, we find again that the local magnetization $M$ vanish suddenly when the virtual bond dimension $D \geq 8$. The result also prove that the ground state in the $S=2$ case have no magnetic.

\begin{figure}[H]
	\centering
	\includegraphics[width=0.75\textwidth]{figures/3pess_S2GE.png}
	\caption[The Energy per site of the $S=2$ Heisenberg model as a function of the virtual bond dimension $D$]{The Energy per site of the $S=2$ Heisenberg model as a function of the virtual bond dimension $D$.}
	\label{fig4330}
\end{figure}

\begin{figure}[H]
	\centering
	\includegraphics[width=0.75\textwidth]{figures/3pess_S2M.png}
	\caption[The trimerization parameter, $\Delta E = 2 |E_{\Delta}-E_{\nabla}| / 3$, and the local magnetization $M$ of the $S=1$ Heisenberg model as functions of the virtual bond dimension $D$]{The trimerization parameter, $\Delta E = 2 |E_{\Delta}-E_{\nabla}| / 3$, and the local magnetization $M$ of the $S=1$ Heisenberg model as functions of the virtual bond dimension $D$.}
	\label{fig4331}
\end{figure}

\subsection{4-PESS for Heisenberg model on square lattice}

For the $spin- \frac{1}{2}$ Heisenberg model on square lattice, we also find that the accuracy is strongly dependent on the virtual bond dimension $D$, see Fig.~\ref{fig4332}. However, the 4-PESS ansatz is unstable with some specific dimension $D$, such as, 6,9, etc. In these cases, the algorithm might hard converge or even be broken. 

Our comparison among two dimensional algorithms is shown in Fig.~\ref{fig4333}. The algorithms started from $D=2$ and setted the cutoff $\varepsilon = 10^{-7}$ to determine how many basis should be truncated. The results shows that the ground states obtain by 4-PESS ansatz is more accuracy then by 2D-iTEBD and the dimension of projection tensors is less than in PEPS.

\begin{figure}[H]
	\centering
	\includegraphics[width=0.75\textwidth]{figures/4pess_HeiGE.png}
	\caption[The Energy per site of the $S=\frac{1}{2}$ Heisenberg model on the square lattice as a function of the virtual bond dimension $D$]{The Energy per site of the $S=\frac{1}{2}$ Heisenberg model on the square lattice as a function of the virtual bond dimension $D$}
	\label{fig4332}
\end{figure}

\begin{figure}[H]
	\centering
	\includegraphics[width=0.75\textwidth]{figures/compare_4pess.jpeg}
	\caption[Compare the per epoch energy of Heisenberg model on two-dimensional square lattice and the requirement of the virtaul bond dimension as the cutoff of the truncation error $\varepsilon = 10^{-7}$ among 2D-iTEBD-like and 4-PESS algorithms]{Compare the per epoch energy of Heisenberg model on two-dimensional square lattice and the requirement of the virtual bond dimension as the cutoff of the truncation error $\varepsilon = 10^{-7}$ among 2D-iTEBD-like and 4-PESS algorithms.}
	\label{fig4333}
\end{figure}
