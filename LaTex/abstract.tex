%!TEX root = thesis.tex \begin{abstractzh}
\providecommand{\keywords}[1]{\textbf{\textit{Key words---}} #1}

\begin{abstractzh}
如何判斷多體量子系統的相變化,且從微觀系統來得到巨觀上的物理性質,在現代仍為凝態物理學中十分有趣的領域。

從NRG的為起點開始,多年來出現了許多突破性的演算法。其中DMRG在一維的系統的模擬中得到了相當好的結果。但在二微系統中,因為 Area law 的關係使其表現不如在一微系統中精確,不僅如此,在二維系統中,計算複雜度上升之速度也非一維系統可比擬。為了解決這些問題,因而出現了許許多多不同的建立在張亮網路理論上的演算法。

此篇論文,紀錄了幾個當今較為主流或新穎並用以模疑二維量子系統的張亮網路演算法。一開始將簡單解釋張亮網路的基本理論; 再來會介紹如何實做、優化演算法,以增加精確度和降低計算複雜度。章節中也附上偽代碼,來說明實作中應注意之細節。最後會比較它們計算二維易辛模型與海森堡模型的結果,來說明各演算法之優缺點。

最後,會附上以 Uni10 建立的開放原始碼,已方便使用者學習與操作。
\end{abstractzh}

\begin{abstracten}
matrix product state(MPS), projected entangled pair state(PEPS), projected entangled simplex state, infinite time-evolveing block-decimation, corner transfer matrix, tensor renormalization group, lBenchmarks, uni10.
\end{abstracten}

\keywords{matrix product state(MPS), projected entangled pair state(PEPS), projected entangled simplex state, infinite time-evolveing block-decimation, corner transfer matrix, tensor renormalization group, lBenchmarks, uni10.}

\begin{comment}
\category{I2.10}{Computing Methodologies}{Artificial Intelligence --
Vision and Scene Understanding} \category{H5.3}{Information
Systems}{Information Interfaces and Presentation (HCI) -- Web-based
Interaction.}

\terms{Design, Human factors, Performance.}
\end{comment}
