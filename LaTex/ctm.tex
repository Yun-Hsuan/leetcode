%!TEX root = thesis.tex
\chapter{Corner Transfer Matrix}
\label{chapter:ctm}
The \textit{cerner transfer matrix renormalization group} (CTMRG) \cite{doi:10.1143/JPSJ.65.891,PhysRevB.80.094403,PhysRevB.84.041108} is an algorithm to numerically compute the \textit{effective environments} which is an approximation of the environment of systems. For example, if the infinite PEPS is composed by a single tensor $A^{h}_{uldr}$ repeatedly, where $h$ express a physical basis of  $\mathbb{V}$ with dimension $d$, and $u,l,d,r$ are virtual bonds with dimension $D$, see Fig.~\ref{fig501}(a). Then we can represent the scale norm $\Braket{\psi|\psi}$ by a simple two dimensional tensor network $\varepsilon$ which is characterized by reduced tensors $a$, as shown in Fig.~\ref{fig501}(b). The reduced tensor $a$ is defined as Eq.~(\ref{reduce_a})
\begin{align}
	\label{reduce_a}
	a \equiv \sum_{h=1}^{d} A_{h} \otimes A^{*}_{h}.
\end{align}
Then, see Fig.~\ref{fig501}(c-d), the environment $\varepsilon^{[\vec{r}]}$ can be regarded as a composite of reduced tensors surrounded the site $\vec{r}$. Finally, to simulate the \textit{effective environments} $G^{[\vec{r}]}$ is the goal of the CTMRG.

In the following sections, we will show more details of implementation of CTM and compare the features between obtaining the states from iPEPE and PESS.

\begin{figure}[ht]
	\centering
	\includegraphics[width=0.75\textwidth]{figures/fig501.png}
	\caption[The tensor diagrams of the corner transfer matrix with the one-site unit cell.]{The tensor diagrams of corner transfer matrix. (a) The reduced tensor $a$ is obtained from the iPEPS state $A$, where $a \equiv \sum_{h=1}^{d} A_{h} \otimes A^{*}_{h}$. (b) An infinite two-dimensional tensor network $\varepsilon$ . (c) The environment $\varepsilon^{[\vec{r}]}$of the site $\vec{r}$. (d) The effective environment $G^{[\vec{r}]}$.}
	\label{fig501}
\end{figure}

\section{Obtain States from PEPS}
\label{2ditebdctm}
In Chap.\ref{chapter:2ditebd}, we have discussed about obtaining the infinite PEPS state $\Ket{\psi}$ of an infinite 2D square lattice by 2D-iTEBD and known that the infinite PEPS could be characterized by two tensors $A^h_{uldr}$ and $B^h_{drul}$ repeatedly [Fig.~\ref{fig511}(a)]. The scalar norm of the iPEPS $\Braket{\psi|\psi}$ is composed of reduced tensors $a_{\bar{u}\bar{l}\bar{d}\bar{r}}$ and $b_{\bar{d}\bar{r}\bar{u}\bar{l}}$ [Fig.~\ref{fig511}(b)]
\begin{align}
	\bar{a} \equiv \sum_{h=1}^{d} A_{h} \otimes A^{*}_{h} \\
	\label{reduce_b}
	\bar{b} \equiv \sum_{h=1}^{d} B_{h} \otimes B^{*}_{h}
\end{align}.
Hence, the environment $\varepsilon^{\left[\vec{r_1},\vec{r_2},\vec{r_3},\vec{r_4}\right]}$ can be regarded as a composite of reduced tensors surrounded a four-site unit cell [Fig.~\ref{fig511}(c)], and the effective environment $G^{\left[\vec{r_1},\vec{r_2},\vec{r_3},\vec{r_4}\right]}$ [Fig.~\ref{fig511}(d)] consists of $C_1, T_{a1}, T_{b1},C_2, T_{a2}, T_{b2},C_3, T_{a3}, T_{b3},C_4, T_{a4}, T_{b4},$

\begin{figure}[ht]
	\centering
	\includegraphics[width=0.75\textwidth]{figures/fig511.png}
	\caption[The tensor diagrams of the corner transfer matrix with the four-site unit cell.]{The tensor diagrams of the corner transfer matrix with the four-site unit cell.(a) The reduced tensor $a$ and $b$ are obtained from the iPEPS state $A$ and $B$, where $a \equiv \sum_{h=1}^{d} A_{h} \otimes A^{*}_{h}$ and $b \equiv \sum_{h=1}^{d} B_{h} \otimes B^{*}_{h}$. (b) An infinite two-dimensional tensor network $\varepsilon$. (c) The environment of the four-site unit cell. (d) The effective environment $G^{\left[\vec{r_1},\vec{r_2},\vec{r_3},\vec{r_4}\right]} = \{ C_1, T_{a1}, T_{b1},C_2, T_{a2}, T_{b2},C_3, T_{a3}, T_{b3},C_4, T_{a4}, T_{b4}\}$.}
	\label{fig511}
\end{figure}

%For the approximation of environment, the directional variant of the CTMRG was developed. 
According to \textit{directional coarse-graining moves}, the effective environment could be updated from four different moves, left, right, up and down and iterated until the environments converges. 

For instance, the procedures to the left move, shown in the Fig.~\ref{fig512} which is derived by Roman and Vidal, is made up of four major steps,
\begin{enumerate}
	\item Insertion: Insert two new columns which consist of \{ $T_{a1},b,a,T_{b3}$ \} and \{ $T_{b1},a,b,T_{a3}$ \} as in Fig.~\ref{fig512}(b).
	\item Absorption: In order to obtaining two new corner matrices $\tilde{C_1}$ and $\tilde{C_4}$, and two new transfer matrices $\tilde{T_{b4}}$ and $\tilde{T_{a4}}$, we contract tensors $C_1$ and $T_{b1}$, tensors $C_3$ and $T_{a3}$, tensors $T_{a4}$ and $b$, and tensors $T_{b4}$ and $a$. Then, contract tensors $\tilde{C_1}$ and $\tilde{T_{b4}}$, and tensors $\tilde{C_4}$ and $\tilde{T_{a4}}$, obtaining $\tilde{Q_1}$ and $\tilde{Q_4}$ which play significant rules for calculating isometries between $\tilde{T_{b4}}$ and $\tilde{T_{a4}}$ as in Fig.~\ref{fig512}(c).
	\item Renormalization: Truncate the vertical virtual bonds of $\widetilde{C_1}$, $\widetilde{T_{b4}}$, $\widetilde{T_{a4}}$, and $\widetilde{C_4}$ by contracting the isometries $Z$ and $W$, where
\begin{align}
	\label{isometry}
	&Z^{\dagger}Z = I \\
	&W^{\dagger}W = I
\end{align}
and the renormalization of the left CTM, yield as
\begin{align}
	\label{renormalize}
	&C^{\prime}_1 = Z^{\dagger} \tilde{C_1} \\
	&T_{b4}^{\prime} = Z\tilde{T_{b4}}W^{\dagger} \\
	&T_{a4}^{\prime} = W\tilde{T_{b4}}Z^{\dagger} \\
	&C^{\prime}_4 = Z\widetilde{C_4}
\end{align}
See the Fig.~\ref{fig512}(d) and \ref{fig512}(f).
	\item Truncation: To determinate the isometries Z and W in the \textit{renormalization} steps is the most significant part. In this case, we use the eigenvalue decomposition of 
		\begin{align}
			\label{eigh_ctm}
			&\tilde{C_1}\tilde{C^{\dagger}_1}+\tilde{C_4}\tilde{C^{\dagger}_4}= \tilde{Z} D_z \tilde{Z^{\dagger}}\\
			&\tilde{Q_1}\tilde{Q^{\dagger}_1}+\tilde{Q_4}\tilde{Q^{\dagger}_4}= \tilde{W} D_w \tilde{W^{\dagger}}
		\end{align}
		shown in Fig.~\ref{fig512}(e). It's not hard to find that the 
		the dimension of bonds of $D_z$ and $D_w$ increase to $\chi^2$. For that reason, we have to truncate $\tilde{Z}$ and $\tilde{W}$ to isometries $Z$ and $W$ which are equivalent to keeping the columns of $\tilde{Z}$ and $\tilde{W}$ corresponding to $\chi$ largest eigenvalues of $D_z$ and $D_w$.
\end{enumerate}

Now, we need to repeat the procedures in Fig.~\ref{fig512}(b)-(d) again for absorbing the other inserted column and obtain a new effective environment $G^{\prime \left[\vec{r_1},\vec{r_2},\vec{r_3},\vec{r_4}\right]}$. By composing four variant moves of the CTM we build an epoch of CTMRG. 

\begin{figure}[ht]
	\centering
	\includegraphics[width=0.80\textwidth]{figures/fig512.png}
	\includegraphics[width=0.80\textwidth]{figures/fig513.png}
	\caption[The procedures of the corner transfer matrix.]{The procedures of the corner transfer matrix. The more discussions are derived in the paragraph}
	\label{fig512}
\end{figure}

\section{Obtain States from PESS}
\label{pessctm}
In this section, we apply the CTM to approximate the effective environment of 4-PESS structure. Firstly, we must transform it to iPEPS structure which is suit to the form of CTM. As shown in Fig.~\ref{fig521}, the projection tensors, $U^{[0]}$, $U^{[1]}$, $U^{[2]}$ and $U^{[3]}$, and the entangled simplex tensors, $S^{[\alpha]}$ and $S^{[\beta]}$ are obtained from 4-PESS ansatz. To map these states to PEPS-like structure, we group the tensors, $S^{[\alpha]}$, $U^{[0]}$ and $U^{[1]}$, in red rectangles into the tensor $A$, 
\begin{align}
	A^{\sigma_i \sigma_j}_{i^{\prime}j^{\prime}kl} = \sum_{ij}{U^{[0]}_{ ii^{\prime},\sigma_i} S^{[\alpha]}_{ijkl} U^{[1]}_{ jj^{\prime},\sigma_j}}
\end{align}
and group, $S^{[\beta]}$, $U^{[2]}$ and $U^{[3]}$, in blue rectangles into the tensor $B$
\begin{align}
	B^{\sigma_k \sigma_l}_{i^{\prime}j^{\prime}kl} = \sum_{k^{\prime}l^{\prime}}{U^{[2]}_{ kk^{\prime},\sigma_i} S^{[\beta]}_{i^{\prime}j^{\prime}k^{\prime}l^{\prime}} U^{[3]}_{ ll^{\prime},\sigma_j}}
\end{align}
, whee the ranks of tensor $A$ and $B$ are six and there are two physical bonds contained in each of them. Hence, after combined the physical bonds in tensors $A$ and $B$, the iPEPS structure will be obtained,
\begin{align}
	A^{\sigma_i \sigma_j}_{i^{\prime}j^{\prime}kl} \rightarrow  A^{\sigma_A}_{i^{\prime}j^{\prime}kl} \\
	B^{\sigma_k \sigma_l}_{i^{\prime}j^{\prime}kl} \rightarrow  B^{\sigma_B}_{i^{\prime}j^{\prime}kl}
\end{align}
Next, in order to make the structure more balance, the entanglement should be well-distributed between each sites, 
\begin{align}
	\widetilde{A} = \sum_{i^{\prime}j^{\prime}kl}{\lambda^{[\beta]^{\frac{1}{2}}}_{i^{\prime}} \lambda^{[\beta]^{\frac{1}{2}}}_{j^{\prime}} A^{\sigma_A}_{i^{\prime}j^{\prime}kl}\lambda^{[\alpha]-\frac{1}{2}}_{l} \lambda^{[\alpha]^{-\frac{1}{2}}}_{k}}\\
	\widetilde{B} = \sum_{i^{\prime}j^{\prime}kl}{\lambda^{[\alpha]^{\frac{1}{2}}}_{k} \lambda^{[\alpha]^{\frac{1}{2}}}_{l} B^{\sigma_B}_{i^{\prime}j^{\prime}kl} \lambda^{[\beta]^{-\frac{1}{2}}}_{i} \lambda^{[\beta]^{-\frac{1}{2}}}_{j^{\prime}}}
\end{align}
and substitute $\widetilde{A}$ and $\widetilde{B}$ into Eq. \ref{reduce_a} and Eq. \ref{reduce_b} to obtain reduced tensors $a$ and $b$. In the end, we apply these two reduced tensor to build the form of CTM and follow the procedures shown in Fig.~\ref{fig512} to simulate the effective environment tensors.

\begin{figure}[ht]
	\centering
	\includegraphics[width=0.70\textwidth]{figures/fig521.png}
	\caption[The tensor diagram of obtaining the reduce tensors from 4-PESS structure.]{The tensor diagram of obtaining the reduce tensors from 4-PESS structure. The reduce tensor $a$ and $b$ are composed by the tensors in the red and blue rectangles}
	\label{fig521}
\end{figure}

\section{Comparison}

To compare the performance of the approximations, we have applied 2D-iTEBD and PESS to approach the ground state of the spin-1/2 quantum transverse Ising model, 
\begin{align}
	H = -\sum_{<\vec{r},\vec{r}^{\prime}>}{\sigma_z^{[ \vec{r}^{} ]} \sigma_z^{[ \vec{r}^{\prime} ]}} - \lambda \sum_{\vec{r}}{\sigma_x^{[ \vec{r} ]}}
\end{align}
, and use directional CTM to obtain the effective environment. See Fig.~\ref{fig522}, the order-parameter $m_z \equiv  \Bra{\Psi} \sigma_z \Ket{\Psi}$ as a function of the external magnetic field $\lambda$. We find that when measuring the local observable with directional CTM, the better ground states are obtained. However, it have no improvement when approaching to near-critical point. The possible reason is that 
the original states computed by iPEPS approximation is not accurate sufficiently. Next, turn to the cases which states are obtained from 4-PESS algorithm.When $D=2$ and $\chi = 20$, we find that it is hard to converge near the critical point because the virtual bonds dimension is too small to describe the system. After increasing the virtual dimension, we notice that it converges to $\lambda_c \approx 3.220$. In Sec.~\ref{chapter:ipess}, we show that the ground states obtained by 4-PESS is more accurate. Hence, it is not astonish that the result compute with 4-PESS+CTM is better than 2D-iTEBD+CTM. However, it still can not compare with the quantum Monte Carlo estimation which show that $\lambda^{MC}_c \approx 3.044$ \cite{PhysRevE.66.066110}.

\begin{figure}[H]
	\centering
	\includegraphics[width=0.75\textwidth]{figures/ctm_itebd.png}
	\caption[Compare the order-parameter $m_z$ of the transfer Ising model on square lattice between 2D-iTEBD and 2D-iTEBD+CTM.]{Compare the order-parameter $m_z$ of the transfer Ising model on square lattice between 2D-iTEBD and 2D-iTEBD+CTM, where the order-parameter $m_z$ as a function of the external field $\lambda$}
	\label{fig522}
\end{figure}

\begin{figure}[H]
	\centering
	\includegraphics[width=0.75\textwidth]{figures/ctm_pess.png}
	\caption[Compare the order-parameter $m_z$ of the transfer Ising model on square lattice between 2D-iTEBD+CTM and 4-PESS+CTM.]{Compare the order-parameter $m_z$ of the transfer Ising model on square lattice between 2D-iTEBD+CTM and 4-PESS+CTM, where the order-parameter $m_z$ as a function of the external field $\lambda$}
	\label{fig523}
\end{figure}


