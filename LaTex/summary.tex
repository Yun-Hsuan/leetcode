%!TEX root = thesis.tex
\chapter{Summary}
\label{chapter:summary}
In one dimensional system, the performance of 1D-iTEBD is pretty well, because it obey the canonical form and have less influences of environment. However, in 2-D systems, we need consider the environment more restrictively when measuring the local observable due to the area law. Moreover, the computational consumption is another serious problem, owing to the growth of a state's dimension which is proportional to $dD^4$. Hence, more and more algorithms and optimizations were developed.

So far, the concept of 2D-iTEBD is still widely applied to study the condense matter physics due to its high efficiency...
