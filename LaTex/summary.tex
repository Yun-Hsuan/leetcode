%!TEX root = thesis.tex
\chapter{Summary}
\label{chapter:summary}

In this thesis, we reviewed the concept of matrix product states (MPS) and drew the structure with tensor diagrams, where the virtual bond dimension $\chi$ between each spins represent that how many the basis and the entanglement information are kept. Next, we introduced the imaginary time-evolving block decimation algorithm (iTEBD), which is the most simple tool to obtain the ground states of the MPS structure. In one dimensional system, the performance of iTEBD have been proved stable and efficient, because it obey the canonical form and have less influences of environment. Then, owing to the success of 1D-iTEBD, we tried to utilize it to simulate the two dimensional systems. However, we encountered in some problems. Firstly, due to the area law, we need consider the environment more restrictively when measuring the local observable. Secondly, the growth of computational consumption to describe project entangled pair states (PEPS) is too large because the dimension of each states is proportional to $dD^4$, where $D$ is the dimension of virtual bonds in PEPS.

Therefore, optimizing two-dimensional algorithms become an significant work. In the Sec.~\ref{chapter:2ditebd}, we started from basic simple update which is unstable due to multiplying to many pseudo-inverse entangled matrices. Next, to improve the stability of 2D-iTEBD, the new simulation proceeds was developed by Hastings. However, these two methods are not useful to study two-dimensional systems with large bond dimensions because the dimension of the projected tensor $\Theta$ is $d^2D^6$ and the cost CPU time will grow exponentially. Hence, we had better applied the decomposition tools, LQ and RQ, to reduce the dimension of the tensor $\Theta$ from $d^2D^6$ to $d^4D^2$ and it will improve the efficiency effectively. Finally, we have noticed that the ways to initialize the states and setting a suitable cutoff $\varepsilon$ to determined how many basis should be truncated have a certain impact on the accuracy and stability of the algorithms. 

Since the PEPS structure is hard to describe the interactions between next-neighbor states. In Sec.~\ref{chapter:ipess}, we have introduced project entangled simplex state (PESS) ansatz to obtain the ground states in two-dimensional systems. Instead of containing the entangled information between each sites, we applied $n$-rank tensors to describe the entanglement in simplices. In conclusion, the computational consumption is less than PEPS ansatz because the dimension of the states on each sites is reduced to $dD^2$ and can obtain the ground states more accuracy in strong-correlated and frustrated systems, such as kagome and Husimi lattices. However, in square lattice systems, the PESS ansatz is not only hard to converge but also unstable and even broken in the end.

Finally, we reviewed the corner transfer matrix (CTM) to consider the influences of the environment. In conclusion, the accuracy will be improved when we measure the local observable with effective environment. However, so far we still can not simulate the environment with large virtual bond dimension $D$ simply because the dimension of the reduce tensors is proportional to $D^8$, which means that the consumption and the cost time would increase exponentially. To deal the obstacle, our lab have developed the open source, Uni10, which not only make the implementation of tensor network algorithms conveniently but also can easily accelerate with GPU.

