%!TEX root = thesis.tex
\chapter{Summary}
\label{chapter:summary}
To study phase transitions and critical behaviors, Monte Carlo methods and finite-size scaling are necessary. Caused by the irrelevant field of finite size, there is some correction depending on size.
To get precise results, large-size simulations should be performed.

To simulate large size system in less time, we implement the Monte Carlo method on GPUs.
According to the checkerboard algorithm, We can parallelize the singe-spin flips and measurements into threads on hundreds of GPU cores.
It does reduce the time for Monte Carlo simulations.

We study the simple XY model. With no anisotropy, we obtain $\nu = 0.6708(1)$ and $\gamma=1.324(4)$, consistent with previous results.
The emergent U(1) symmetry is a characteristic in the transitions of anisotropic XY model.
Following the work of \citeauthor{lou_emergence_2007}, we study the the length scale $\Lambda$ of U(1) symmetry with the Monte Carlo method on GPU.
For the  dangerous irrelevance, $\Lambda$ diverge faster than the correlation length, $\Lambda\sim \xi^{a_q},\ a=\nu_q/\nu>1$.
For the simple XY model with $Z_q$ anisotropy, we find $a_4=1.06(4)$, $a_5=1.6(1)$, $a_6=2.4(1)$, $a_6=2.4(1)$, $a_7=3.0(1)$, $a_8=4.4(3)$, which are consistent with previous works.

For the XY model with nonlinear potential, the transitions are first-order for $P$ large enough.
For $P>1$, with $Z_q$ anisotropy, $\nu_q$ and $a_q$ vary with $h$, the strength of anisotropy. And, $\nu_q$ may be smaller than $\nu$ for $h=0$.
In the case with $Z_4$ anisotropy, for $P\geq1.5$, the transitions turn into first-order transition for $h$ large enough, and as $h$ increasing, the exponents become non-universal.
For $q>4$, there are two different cases depending on $P$.
For $P$ large enough, the critical behaviors and the behavior of $\Lambda$ are almost the same as the $Z_4$ case, but the non-universal behaviors can not be confirmed clearly because of errors caused by finite-size effects.
In the case that $P$ is not so large, the transition is always continuous.
Though $\nu_q$ also vary with $h$, it is always larger than $\nu$ for $h=0$.
The anisotropy should be irrelevant, but it is not sure that the critical exponents are constant with different $h$.

An important question left is the regions of non-universal behavior.
To answer this question, simulations on systems with larger sizes should be carried out.