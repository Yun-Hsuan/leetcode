%!TEX root = thesis.tex
\chapter{Tensor Network Theory}
In tensor network theory, we are used to represent tensors graphically instead of complicated equations, because \textit{tensor diagrams} can map to quantum states and geometric lattices explicitly. Base on its clearly representation, we can implement tensor network algorithms simply. This section begins from a fundamental question: How to draw a tensor network diagrams?

\section{Representation of tensors in tensor Networks}
\label{notations}
In mathematical concept, a tensor is considered as a multi-dimensional array of scalers. The arrangement of the elements in a tensor is dependent on its \textit{indices} and the \textit{rank} of tensor is equivalent to the number of indices. Thus, a rank-0 tensor is a scaler $(T)$, a rank-1 tensor is a vector $(T_{i})$, a rank-2 tensor is a matrix $(T_{ij})$ and so on.



\section{Tensor operations and tensor network diagrams} % (fold)
\label{operation}

\subsection{Reducing Computational Complexity} % (fold)
\label{sub:reduce}

Make simulation possible and improve the accuracy and efficiency.
